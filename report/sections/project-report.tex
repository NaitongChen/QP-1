% !TEX root = ../main.tex

% Project report section

\section{Project report}

In this section, we compare the model selection procedure for ridge regression (i.e. selecting the regularization parameter) using SURE as the objective against those based on k-fold cross-validation. We also include ordinary least squares (OLS) in our comparison as a baseline, representing the case of no regularization ($\lambda=0$). The Python code used to run the experiments and generate the figures can be found at \url{https://github.com/NaitongChen/QP-1}.

$ $\newline
We begin by formulating both the SURE and k-fold cross-validation model selection procedures. Recall in our setting of a linear regression problem, we have $y\sim\distNorm\left( X\beta, \sigma^2I \right)$, where $X\in\reals^{n\times p}$, $y\in\reals^n$, $\sigma>0$. Also recall that we assume the data are centred and so an intercept term is not needed. We know that for $\lambda>0$, the ridge estimate of the regression coefficients take on the form
\[
\hat{\beta}_{\text{ridge}} = \left( X^TX + \lambda I \right)^{-1}X^Ty,
\] 
then we can write our fitted values as
\[
\hy_\lambda(y) &= X\hat{\beta}_{\text{ridge}} = X\left( X^TX + \lambda I \right)^{-1}X^Ty.
\]
Under this setup, the divergence term can be written as
\[
\sum_{i=1}^n \frac{\hy_{\lambda, i}(y)}{\partial y_i} = \sum_{i=1}^n \frac{\partial}{\partial y_i} \left ( X_i^T \left( X^TX + \lambda I \right)^{-1}X^Ty \right) = \tr\left( X\left( X^TX + \lambda I \right)^{-1}X^T \right) = \sum_{j=1}^p \frac{d_j^2}{d_j^2 + \lambda},
\]
where the last term is obtained through the singular value decomposition $X = UDV^T$, where $D$ is a diagonal matrix with the singular values $\begin{bmatrix} d_1 & \cdots & d_p \end{bmatrix}$ on the diagonal. We can now write our SURE as
\[
\hat{R}(\lambda) &= -n\sigma^2 + \| y - \hy_\lambda(y) \|_2^2 + 2\sigma^2 \sum_{i=1}^n \frac{\hy_{\lambda, i}(y)}{\partial y_i}\\
&= -n\sigma^2 + \| y - \hy_\lambda(y) \|_2^2 + 2\sigma^2 \sum_{j=1}^p \frac{d_j^2}{d_j^2 + \lambda}\\
&= -n\sigma^2 + \| y - \hy_\lambda(y) \|_2^2 + 2\sigma^2 \text{edf}(\lambda).
\]
Note that $\text{edf}(\lambda)$, the effective degrees of freedom, characterizes the complexity of the model. As $\lambda$ increases, while we reduce the complexity of the model, the sum of squares residual error reflected through $\| y - \hy_\lambda(y) \|_2^2$ will increase. The SURE then reflects this balance of the bias-variance trade-off. To select $\lambda$, we can minimize $\hat{R}$ over $\lambda$ using gradient descent and automatic differentiation. Note that since we require $\lambda>0$, we work in the unconstrained parameter space by applying a $\log$ transformation to $\lambda$.

$ $\newline
For k-fold cross-validation, we begin by dividing the dataset into $k$ (almost) equal parts of size $N_1,\dots, N_k$ s.t. $\sum_{i=1}^k N_i = N$. We denote the index set of each fold as $\mcI_{N_1}, \dots, \mcI_{N_k}$. Given each fold, we compute the mean square prediction error (MSPE) using the regression coefficients estimated with data from all other folds. We pick the regularization parameter $\lambda$ that minimizes the sum of MSPEs across all folds. We can write the k-fold cross-validation procedure as
\[
L(\lambda) = \sum_{i=1}^k \frac{1}{N_i} \sum_{j \in \mcI_{N_i}} \left(y_j - x_j^T\hbeta_{\text{ridge}, j}(\lambda)\right)^2,
\] 
where
\[
\hbeta_{\text{ridge}, j}(\lambda) = \argmin_{\beta\in\reals^p} \frac{1}{N-N_j}\sum_{l\notin\mcI_{N_j}}(x_l^T\beta - y_l)^2 + \lambda \|\beta\|_2^2.
\]
Note that for each $\lambda$, evaluating the loss requires fitting $k$ ridge regression models. In the special case where $k=N$, namely leave-one-out cross-validation (LOOCV), the above loss simplifies to
\[
L_{\text{LOOCV}}(\lambda) = \frac{1}{N}\sum_{i=1}^N \left( \frac{y_i - \hbeta_{\text{ridge}}(\lambda)}{1 - H_{\lambda,i}} \right)^2,
\]
where $\hbeta_{\text{ridge}}$ is the ridge regression estimate using the entire dataset, and $H_{\lambda,i}$ is the $i^\text{th}$ diagonal entry of $H_\lambda = X(X^TX + \lambda I)^{-1}X^T$.

$ $\newline
While minimizing $L(\lambda)$ using automatic differentiation and gradient descent is feasible, except for LOOCV, for each optimization iteration, we are required to fit $k$ ridge regression models, making this procedure computationally expensive. As a result, we follow the standard approach of selecting $\lambda$ over a grid of values. It is also important to note that for $k<N$, the k-fold cross-validation procedure is random over the fold assignment.

$ $\newline
Recall that ridge regression is developed as a method to estimate the coefficients in a linear regression problem where the predictor variables are highly correlated. More specifically, by introducing the regularization parameter $\lambda$, we sacrifice the unbiasedness of the least squares solution in order to reduce the variance of the estimator of the regression coefficients. As a result, we use a synthetic regression problem with highly correlated predictor variables to evaluate the performance of the model selected using SURE compared to k-fold CV and OLS.

$ $\newline
We generate the data for our synthetic regression problem ($p=5$) consisting of $n=100$ observations as described below. For each row of the data matrix $\tdX$, we generate
\[
\begin{bmatrix} \tdX_1 \\ \tdX_2 \\ \tdX_3 \end{bmatrix} \sim \distNorm \left( \begin{bmatrix} -5 \\ 0 \\ 3 \end{bmatrix}, I \right), \quad \tdX_4 = -5\tdX_2 + \distNorm(0, 0.1), \quad \tdX_5 = \tdX_3 + \distNorm(0, 0.1)
\]
We standardize the features so that each column has mean $0$ and variance $1$ and denote this transformed data matrix $X$. The response vector $y$ are then generated by
\[
y = X\begin{bmatrix} 0 & 3 & -1 & 1 & 2 \end{bmatrix}^T + \eps, \quad \text{where} \quad \eps \sim \distNorm(0, I).
\]

%\textbf{SURE with Ridge Regression:}
%
%Let $y\sim\distNorm\left( X^T\beta, \sigma^2 \right)$, where $y\in\reals$ and $X\in\reals^{p+1}$, $X$ constant. Then we have $y\sim\distNorm\left( X\beta, \sigma^2I \right)$, where $\biy\in\reals^n$ and $\biX\in\reals^{n\times p}$.
%
%We know that $\hat{\beta}_{\text{ridge}} = \left( X^TX + \lambda I_{p} \right)^{-1}X^Ty$, then 
%\[
%\hat{\mu}_\lambda(\biy) &= \biX\hat{\beta}_{\text{ridge}} = \biX\left( \biX^T\biX + \lambda I_{p+1} \right)^{-1}\biX^T\biy\\
%\hat{\mu}_{\lambda, i}(\biy) &= X_i^T\hat{\beta}_{\text{ridge}} = X_i^T\left( \biX^T\biX + \lambda I_{p+1} \right)^{-1}\biX^T\biy
%\]
%Then
%\[
%\frac{\hat{\mu}_{\lambda, i}(\biy)}{\partial y_i} &= \frac{\partial}{\partial y_i} \left ( X_i^T \left( \biX^T\biX + \lambda I_{p+1} \right)^{-1}\biX^T\biy \right)\\
%&= \frac{\partial}{\partial y_i} F_i \biy \quad\quad (F_i \defas X_i^T\left( \biX^T\biX + \lambda I_{p+1} \right)^{-1}\biX^T \in \reals^{n})\\
%&= F_{i,i}\\
%&= \left( X_i^T\left( \biX^T\biX + \lambda I_{p+1} \right)^{-1}\biX^T \right)_i.
%\]
%We can now write
%\[
%\hat{R} &= -n\sigma^2 + \| \biy - \hat{\mu}_\lambda(\biy) \|_2^2 + 2\sigma^2 \sum_{i=1}^n \left( X_i^T\left( \biX^T\biX + \lambda I_{p+1} \right)^{-1}\biX^T \right)_i\\
%&= -n\sigma^2 + \| \biy - \hat{\mu}_\lambda(\biy) \|_2^2 + 2\sigma^2 \tr\left( \biX\left( \biX^T\biX + \lambda I_{p+1} \right)^{-1}\biX^T \right)\\
%&= -n\sigma^2 + \| \biy - \hat{\mu}_\lambda(\biy) \|_2^2 + 2\sigma^2 \tr\left( \biX^T\biX\left( \biX^T\biX + \lambda I_{p+1} \right)^{-1} \right)\\
%&= -n\sigma^2 + \| \biy - \hat{\mu}_\lambda(\biy) \|_2^2 + 2\sigma^2 \tr\left( H\left( H + \lambda I_{p+1} \right)^{-1} \right),
%\]
%where the last line is by defining $H\defas\biX^T\biX$. We can optimize $\lambda$ over $\hat{R}$ using autodiff (log-transform $\lambda$ so that it is nonnegative). 